\chapter{Bewegungserkennung}\label{chapter:motion-detection}
In diesem Kapitel sollen die Ergebnisse aus Kapitel \ref{chapter:dataset}
verwendet werden, um eine Bewegungserkennung zu entwickeln. Genauer gesagt soll
auch hier ein Modell entworfen werden, um diese Bewegungen zu erkennen und
klassifizieren. Zudem soll ein Ausblick auf Techniken gegeben werden, um weitere
bewegungsabhängige Eigenschaften mithilfe von KNNs zu extrahieren. Zum Beispiel
ist beim Ausüben von sportlichen Aktivitäten neben der Art auch die Frage, ob
die Bewegung richtig ausgeführt wurde, interessant. Aber auch das Vorhersagen
von zukünftigen Bewegungen anhand kürzlich getätigten Posen kann für einige
Anwendungen hilfreich sein. So auch zum Beispiel beim vorzeitigen Erkennen von
aggressiven Verhaltensmustern, sodass Eingegriffen werden kann, bevor die
kriminelle Tätigkeit ausgeführt werden kann. Da sich diese Arbeit vor allem mit
dem Problem beschäftigen soll, wie solche Modelle auf mobilen Plattformen
überführt werden können, soll schließlich eine Android-App entwickelt werden, um
die Ergebnisse zusammenzufassend zu präsentieren. Android wird als Plattform
gewählt, da es zur Zeit die am häufigsten vertretene mobile Plattform ist und
ein entsprechendes Gerät leicht zur Verfügung steht. Zum Vergleich, Android
besitzt einen Marktanteil von 72,84\%, iOS einen von 26,34\% und 0,82\% werden
von sonstigen Plattformen
gehalten\footnote{https://www.statista.com/statistics/272698/global-market-share-held-by-mobile-operating-systems-since-2009/
(besucht am 13.08.2021)}.

Eine Bewegung ist beschreibbar durch die Änderung des Ortes über die Zeit. Das
bedeutet, dass eine Fotoaufnahme einer sich bewegenden Person nicht ausreicht,
um die Bewegung aufzuzeichnen. Entsprechend müssen Bilder über Zeit aufgenommen
werden. Die nächste Frage, die sich ergibt ist, wie viele Bildaufnahmen nötig
sind, um eine Bewegung erkennen zu können. Dies hängt natürlich von der Art der
Bewegung und der Bewegungsgeschwindigkeit ab. Möchte man beispielsweise einen
Jumping-Jack aufnehmen und die Person bewegt sich viel zu langsam, dann sind
wesentlich mehr Aufnahmen nötig, als wenn diese sich in einem normalen Tempo
bewegt. In der Informationstheorie wird dies mithilfe der Abtastrate
beschrieben, die angibt, wie oft pro Sekunde abgetastet werden soll. Damit in
Verbindung kann man die minimale Rate durch das Niquist-Shannon-Abtasttheorem
bestimmen, welches aussagt, dass ein Signal exakt rekonstruierbar ist, wenn eine
Frequenz mit der doppelten Abtastrate abgetastet wird.
\[
    f_{abtast} = 2 \cdot f_{max}
\]
Ein durchschnittlicher Fahrradfahrer schafft eine Trittfrequenz von maximal 60
Umdrehungen pro Minute während Leistungssportler bis zu 110 Umdrehungen pro
Minute schaffen \cite{smolik}. Das würde bedeuten, dass die Abtastrate
mindestens 2 Hz betragen muss, um das Signal rekonstruierbar aufzuzeichnen.
Betrachtet man nun eine Beinpressbewegung, die ebenfalls mit 60 Tritten pro
Minute ausgeführt wird, ergibt sich ebenfalls eine Abtastfrequenz von 2 Hz. Hier
wird folgendes Problem ersichtlich. Tastet man beide Signale mit 2 Hz ab, so
kann man nicht zwischen Kreisbewegung und Linearbewegung unterscheiden, wenn
sich die abgetasteten Punkte überschneiden. Würde man hingegen mit 3 Hz
abtasten, so wären die Bewegungen eindeutig voneinander unterscheidbar. Dies hat
unter anderem damit zu tun, dass eine lineare Funktion durch zwei Punkte und ein
Kreis stets durch drei sich nicht auf einer Geraden befindenden Punkte
beschreibbar ist. Damit also eine mobile App eine Bewegungserkennung durchführen
kann, ist es wichtig, dass diese mit möglichst vielen Frames pro Sekunde (FPS)
ausgeführt wird, um verschiedenste Bewegungen zu erkennen.

Für die Bewegungserkennung ist auch die Erkennung von menschlichen Posen
wichtig, die Schlüsselpunkte des Körpers aus Bildern extrahiert. Aus diesen
Informationen können bereits einige Eigenschaften einer Pose bzw. Bewegung
abgeleitet oder berechnet werden. So ist zum Beispiel die Berechnung des Winkels
zwischen Oberschenkel und Wade relativ simpel, wenn die entsprechenden
Schlüsselpunkte bekannt sind. Dem Benutzer kann dadurch berechnet werden, ob
eine Übung, die abhängig von diesem Winkel ist, richtig ausgeführt wird oder
nicht. Grundsätzlich werden folgende Methoden zum Erkennen bzw. Analysieren von
Bewegungen in Betracht gezogen. Erstens, es werden die von der mobilen Kamera
aufgenommenen Bilder als Eingabe für einen Bildklassifizierer verwendet, welcher
die Art der Bewegung ausgibt. Zweitens, die aufgenommenen Bilder werden als
Eingabe in einen Pose-Detektor gegeben, welcher zunächst die in dem Eingabebild
enthaltenen Schlüsselpunkte ausgibt. Diese werden dann anschließend in einen
angehängten Prediction-Head gegeben, welcher aus den Schlüsselpunkten z.B. die
Bewegungsart erkennt oder andere Aufgaben übernimmt. Das Ziel dieses Abschnitts
ist es also, die beiden Methoden miteinander zu vergleichen und festzustellen,
welche für eine mobile Anwendung geeignet ist.

\section{Erkennung von menschlichen Posen}
Das Erkennen von menschlichen Posen hat in den letzten Jahren viel Aufmerksamkeit bekommen.

\section{Erkennung von Bewegungsarten}
\section{Erkennung von Anomalien}
\section{Erkennung von Eigenschaften}
\section{Vorhersage von Bewegungen}
\section{Architektur einer mobilen Anwendung}