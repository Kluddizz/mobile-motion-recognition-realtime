\chapter{Einleitung}
Wie können komplexe Machine-Learning-Modelle effizient und in Echtzeit auf
mobilen Geräten wie Smartphones ausgeführt werden? Mit dieser Frage soll sich
diese Arbeit beschäftigen. Speziell wird sich dabei auf die Erkennung und
Analyse von Bewegungen bezogen. Dabei müssen bereits vorhandene Modelle
umgewandelt werden, um auf mobilen Geräten eine Echtzeiterkennung durchführen zu
können. Künstliche Intelligenz hat bereits in vielen verschiedenen Bereichen
eine unterstützende Rolle eingenommen.  Dementsprechend ist das Feld in den
letzten Jahren stetig gewachsen und hat an Interesse gewonnen. Viele Anwendungen
funktionieren nur deshalb, weil sie durch Modelle des Machine-Learnings
unterstützt werden. Vor allem in der Computer-Vision findet diese Technologie
Anwendung. Beispiele hierfür sind Bildklassifizierer und Objekt-Detektoren, die
entsprechend Bilder eine Klasse zuordnen bzw. viele Objekte innerhalb eines
Bildes erkennen. Neben der Bildverarbeitung ist die Erkennung von menschlichen
Posen bzw. von Bewegungen mit künstlichen neuronalen Netzen (KNN) ein weiteres,
aktuelles Forschungsthema. Diese Art von Detektoren werden unter anderem dazu
verwendet, um Schlüs\-sel\-punkte des menschlichen Körpers zu identifizieren.

Während solche Modelle bereits im Desktopbereich mit weniger Einschränkungen
ausgeführt werden können, sind diese eher schwierig auf ressourcenarme Geräte
übertragbar. Oft müssen abgewandelte, verkürzte Varianten erstellt werden, um
die benötigte Rechenleistung so gering wie möglich zu halten -- die meisten
Smartphones haben zur Zeit leider nicht die gleichen Rechen- und
Speicherkapazitäten wie die meisten Desktopmaschinen, ganz zu schweigen von
diversen anderen Geräten des Internet-of-Things (IoT) wie Haushaltsgeräte und
Sensoren. Aus diesem Grund soll sich diese Arbeit insbesondere damit
beschäftigen, wie die Bewegungserkennung auf mobilen Geräten ausgeführt werden
kann. Zusätzlich wird untersucht, welche Anpassungen vorhandene
Machine-Learning-Modelle benötigen, um auf mobile Geräte ausgeführt werden zu
können.

Kapitel \ref{chapter:basics} beschäftigt sich mit den Grundlagen der in dieser
Arbeit verwendeten Technologien. Dabei wird unter anderem darauf eingegangen,
wie Unterschiede zwischen Distributionen gemessen werden können, um damit den
Grundstein für spätere Loss-Funktionen zu schaffen. Diese werden dann vor allem
für das Trainieren von Generative-Adversarial-Networks (GANs) verwendet. Auch
werden in diesem Kapitel einige Grundbausteine zum Entwickeln von sehr tiefen
neuronalen Netzen besprochen. Anschließend wird die Funktionsweise von GANs und
entsprechenden Verlustfunktionen zum Trainieren dieser Netzwerke besprochen.
Zusätzlich werden Probleme der einzelnen Architekturen besprochen und Lösungen
vorgestellt.

Kapitel \ref{chapter:dataset} stellt Methoden vor, die zum Erstellen eines
Datensatzes verwendet werden. Dieser Datensatz wird anschließend verwendet, um
die Bewegungserkennung aus dem nächsten Kapitel zu implementieren und die
Modelle damit zu trainieren. Besonders wird in diesem Kapitel der Aufbau des
Datensatzes erläutert und inwiefern dieser mithilfe von GANs erweitert werden
kann. Insbesondere wird hier versucht, einen kompletten Datensatz mithilfe eines
GANs zu erzeugen, sodass dieses dynamisch anstelle eines echten Datensatzes
verwendet werden kann.

Kapitel \ref{chapter:motion-detection} führt die Bewegungserkennung ein und
vergleicht unter anderem verschiedene Ansätze zum Erkennen von menschlichen
Schlüsselpunkten. Dabei wird auf Single- und Multi-Pose-Detection eingegangen
und mit dessen Hilfe eine neue Netzwerkarchitektur definiert, die in der Lage
ist, eine Folge von menschlichen Posen zu klassifizieren und analysieren.