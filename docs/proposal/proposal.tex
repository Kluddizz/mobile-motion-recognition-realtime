\documentclass[dvipsnames,a4paper,12pt]{article}
\title{Proposal: Automatische Datensatzgenerierung und Bewegungserkennung auf mobilen Plattformen}
\author{Florian Hansen}

\usepackage[ngerman]{babel}
\usepackage{setspace}
\usepackage{pgfgantt}
\usepackage{xcolor}
\usepackage{array}
\usepackage[nohead, nomarginpar, margin=1in, foot=.25in]{geometry}

\newcolumntype{L}[1]{>{\arraybackslash}p{#1}}
\newcolumntype{C}[1]{>{\centering\let\newline\\\arraybackslash\hspace{0pt}}m{#1}}
\newcolumntype{R}[1]{>{\raggedleft\let\newline\\\arraybackslash\hspace{0pt}}m{#1}}

\linespread{1.3}

\begin{document}
  \maketitle

  \section{Forschungsziel} % Welches Rätsel lösen?
  Das Ziel dieser Arbeit soll darin bestehen, dass die Bewegungserkennung
  weiter erforscht werden soll. Die zentrale Frage ist also, wie man mithilfe
  von künstlichen neuronalen Netzen unterschiedliche Arten von Bewegungen auf
  mobilen Plattformen erkennen und auswerten kann. Ein Beispiel für solche
  Bewegungen sind sportliche Aktivitäten wie das Hantelheben, Radfahren und
  Rudern. Damit in Verbindung soll auch gleichzeitig das Thema aufgegriffen
  werden, wie Datensätze für ein solches Netz ausgehoben werden können ohne
  zu viel Zeit investieren zu müssen.

  \section{Hintergrund} % Warum ist die Arbeit relevant?
  Bereits im Forschungsprojekt während meines Master-Studiums habe ich mich
  mit dem Erstellen einer Architektur für eine Bewegungserkennung
  be\-schäf\-tigt und darauf basierend eine App entwickelt. Diese App hat
  jedoch nie den Status erreicht, Bewegungen erkennen zu können, sondern konnte
  Objekte in Echtzeit identifizieren, die mit typischen Bewegungen in
  Verbindung stehen, wie z.B. Schuhe. Dies dient damit als Grundlage für diese
  Forschungsarbeit. Auch wurde in dem Forschungsprojekt klar, dass die
  Aushebung von geeigneten Datensätzen zum Training von
  Machine-Learning-Modellen ein großes Problem darstellt, da dieser
  Arbeitsschritt sehr zeitintensiv im Vergleich zu der eigentlichen Arbeit
  ist. Damit entstand in der Vergangenheit ein zeitliches Defizit für die
  Verfolgung der eigentlichen Forschungsziele. Wir haben dabei herausgefunden, dass
  künstlich erzeugte Datensätze diesen Prozess beschleunigen können und es
  kaum negativen Einfluss auf die Erkennungsrate des trainierten Netzes hat.
  Deshalb interessiere ich mich parallel zur Bewegungserkennung auch für eine
  elegantere Art und Weise, Trainingsdaten künstlich in Massen zu erzeugen, ohne
  dabei das eigentliche Ziel der Forschungsarbeit zu vernachlässigen.

  \section{Literatur} % Zitieren, um andere Arbeiten zu unterstützen
  In der Arbeit von \cite{Rahayfeh2013} wird beschrieben, wie wichtig die
  Bewegungserkennung in der Mensch-Maschninen-Interaktion ist und haben sich
  speziell auf Eye-Tracking und Kopfbewegungen konzentriert. In ihrer Arbeit
  wird ebenfalls erwähnt, dass mehr Forschungsaufwand in dieses Gebiet investiert
  werden sollte, damit die besprochenen Methoden in Echtzeit durchgeführt werden
  können.

  In \cite{Bieshaar2018} wird ein neuronales Netz zum Erkennen von Startbewegungen
  von Radfahrern entworfen. Motivation dieser Arbeit war die Interkonnektivität von
  Verkehrsteilnehmern unterschiedlichster Art, die in Zukunft mög\-lich\-er\-wei\-se vorhanden
  sein könnte. Dabei soll eine situations- und absichtsbedingte Vorraussagung helfen,
  die Koorperation zwischen diesen Verkehrsteilnehmern zu erlauben.

  Auch in \cite{Gao2016} wird die Bewegungserkennung verwendet, um die Milchkuhzucht
  zu über\-wa\-chen. Dabei sollen ungewöhnliche Bewegungen erkannt und analysiert werden,
  ohne dass ein Mensch zur ständigen Überwachung eingesetzt wird.

  In \cite{Cust2019} wird außerdem die Bewegung von sportlichen Aktivitäten behandelt
  und wie für diese Aufgabe ein neuronales Netz entworfen werden kann. Speziell wird
  dort ein systemmatischer Überblick über die Entwicklung und die Performace gegeben.

  \section{Vorgehensweise und Methoden} % Theorie oder Empirie
  In der Master-Thesis soll sich zunächst auf den Entwurf einer automatischen Quelle
  für Trainingsdaten bezogen werden, wofür \textit{Generative Adverserial Networks}
  (GANs) geeignet scheinen. Damit sollen Trainingsdaten für neuronale Netze schnell
  geliefert werden können und vor allem zukünftige Arbeiten beschleunigen. Auch in
  dieser Arbeit soll die Datenquelle verwendet werden, um die Bewegungserkennung
  mit künstlichen neuronalen Netzen auf mobilen Plattformen zu erforschen. Dabei soll
  sich vor allem auf eine möglichst allgemeingültige Lösung konzentriert werden.

  \section{Zeitplan und Gliederung} % Arbeitsplan und Gliederung
  \subsection{Zeitplan}
  \begin{center}
    \begin{tikzpicture}
      \begin{ganttchart}[
          expand chart=\linewidth,
          hgrid, time slot format=isodate,
          time slot unit=day,
          link bulge=3,
          link tolerance=2,
          bar label node/.style={text width=4cm,align=right,font=\scriptsize\raggedleft,anchor=east},
          bar/.append style={fill=MidnightBlue},
          canvas/.append style={name=canvas}
        ]{2021-04-01}{2021-08-31}

        \gantttitle{\textbf{Phase 1:} Entwicklung wichtiger Features}{153} \\
        \gantttitlecalendar{year, month=shortname} \\

        \ganttbar{Einarbeitung in GANs und Alternativen}{2021-04-01}{2021-04-14} \\
        \ganttbar{Entwicklung eines GANs für bewegliche Objekte}{2021-04-15}{2021-04-30} \\
        \ganttbar{Entwicklung eines Modells für Bewegungserkennung}{2021-05-01}{2021-07-01} \\
        \ganttbar{Entwicklung einer Android-App}{2021-07-02}{2021-07-15} \\
        \ganttbar{Schreiben der Arbeit und deren Ergebnisse}{2021-07-16}{2021-08-31} \\

        \ganttlink{elem0}{elem1}
        \ganttlink{elem1}{elem2}
        \ganttlink{elem2}{elem3}
        \ganttlink{elem3}{elem4}

      \end{ganttchart}
      \draw [/pgfgantt/canvas,fill=none]
      ([yshift=-0.5\pgflinewidth]canvas.north west) -- 
      (canvas.north west -| current bounding box.west) |- 
      ([yshift=0.5\pgflinewidth]canvas.south west);
    \end{tikzpicture}
  \end{center}

  \subsection{Gliederung}
  \begin{enumerate}
    \item Einleitung
    \item Motivation
    \item Grundlagen
    \item Künstliche Generierung von Datensätzen
    \item Bewegungserkennung
    \item Entwicklung einer Android-App
    \item Fazit und Ausblick
  \end{enumerate}

  \bibliography{bibliography}
  \bibliographystyle{abbrv}
\end{document}
